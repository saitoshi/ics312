% DOCUMENT FORMATING
\documentclass[12pt]{article}
\usepackage[margin=1in]{geometry}

% PACKAGES
\usepackage{amsmath} % For extended formatting
\usepackage{amssymb} % For math symbols
\usepackage{amsthm} % For proof environment
\usepackage{array} % For tables
\usepackage{enumerate} % For lists
\usepackage{extramarks} % For headers and footers
\usepackage{blindtext}
\usepackage{fancyhdr} % For custom headers
\usepackage{graphicx} % For inserting images
\usepackage{multicol} % For multiple columns
\usepackage{verbatim} % For displaying code
\usepackage{tkz-euclide}
\usepackage{pgfplots}
\newtheorem{theorem}{Theorem}[section]
\newtheorem*{theorem*}{Theorem}
\newtheorem{corollary}{Corollary}[theorem]
\newtheorem{lemma}[theorem]{Lemma}
\usepackage{algorithm}
\usepackage[noend]{algpseudocode}
% SET UP HEADER AND FOOTER
\pagestyle{fancy}
\lhead{\MyCourse} % Top left header
\chead{\MyTopicTitle} % Top center header
\rhead{\MyAssignment} % Top right header
\lfoot{\MyCampus} % Bottom left footer
\cfoot{} % Bottom center footer
\rfoot{\MySemester} % Bottom right footer
\renewcommand\headrulewidth{0.4pt} % Size of the header rule
\renewcommand\footrulewidth{0.4pt} % Size of the footer rule
\newcommand{\MyCourse}{ICS 312}
\newcommand{\MyTopicTitle}{HW 4}
\newcommand{\MyAssignment}{Shin Saito}
\newcommand{\MySemester}{Spring 2021}
\newcommand{\MyCampus}{University of Hawaii at Manoa}
\begin{document}
\subsection*{Exercise 1}
\begin{enumerate}
    \item 2-byte quantities: 8FE0 + B036 
   \begin{center}
        \begin{tabular}{c c c c c}
         c& &c    &    & \\
    &8 & F & E & 0 \\
    + & B & 0 & 3 & 6 \\
    \hline 
    1& 4 & 0 & 1 & 6
    \end{tabular}
   \end{center}
   \begin{itemize}
       \item The carry bit is set to CF = 1. 
       \item The overflow bit is set to OF = 1
       \item Since both 8FE9 and B036 are both negative the expected summation outcome is negative. 
       \item However 4016 is positive number therefore, the overflow bit is set. 
       \item The signed extend representation: 0000 4016
       \item Hexadecimal to Decimal Conversion 
       \begin{align*}
           4 \cdot 16^3 + 1 \cdot 16^1 + 6  \\
           = 16406
       \end{align*}
   \end{itemize}
   \begin{algorithm}
   \caption{macro prints}
   \begin{algorithmic}[1]
    \State mov ax, 03076h ; ax : 03076h
    \State movsx eax, ax ; extend to a signed 4 byte value 0000 4016
    \State call print\underline{ }int ; prints 16406 to the screen
   \end{algorithmic}
   \end{algorithm}
   \item 1-byte: E5 + 0E 
   \begin{center}
       \begin{tabular}{c c c}
       & c & \\
        & E  & 5  \\
        +& 0   & E \\
        \hline 
        & F & 3 
       \end{tabular}
   \end{center}
   \begin{itemize}
       \item The carry bit is set to CF = 0
       \item The overflow bit is set to OF = 0. 
       \item E5 is negative and 0E is positive then there is no expectation of overflow. Also E5 is small negative and 0E is a small positive so the summation should also be in the range. 
       \item Hexadecimal to Decimal Conversion \\
       F3 is a negative so take the complement 
       \begin{align*}
           FF - F3 =\\
           0C + 1 = \\
           0D
       \end{align*}
       \clearpage
       Then convert it 
       \begin{align*}
           0 \cdot 16^1 + 13 \\
           = 13
       \end{align*}
       Negate the outcome so -13. 
       \item Signed-Extended: FFFF FFF3 
   \end{itemize}
      \begin{algorithm}
   \caption{macro prints}
   \begin{algorithmic}[1]
   \State mov al, 0F3h ; points ax to F3
    \State movsx eax, al ; extends F3 to a byte signed FFFF FFF3
    \State call print\underline{ }int ; prints -13
   \end{algorithmic}
   \end{algorithm}
   \item 2-byte quantities: 5243 + 7DBC \\
   \begin{center}
        \begin{tabular}{c c c c c}
    &5 & 2 & 4 & 3 \\
    + & 7 & D & B & C \\
    \hline 
    & C & F & F & F
    \end{tabular}
    \end{center}
    \begin{itemize}
    \item The carry bit is set to CF = 0 
    \item The overflow bit is set to OF = 1
    \item The reason is because since 5243 and 7DBC are both positive, the expecation of the summation is positive. However, CFFF is a negative causing the overflow.
    \item Converting Hexadecimal to Decimal \\
    Since CFFF is negative, take the negation of it by finding the two's complementary 
    \begin{align*}
        FFFF - CFFF = \\
         = 3000 + 1 = 3001 \\
         \to \\
         3 \cdot 16^3 + 1 = \\
         12289 \\
         \text{negate it} -12289
    \end{align*}
    \item sign extended: FFFF CFFF
   \end{itemize}
         \begin{algorithm}
   \caption{macro prints}
   \begin{algorithmic}[1]
   \State mov ax, 0CFh ; move al to CF
    \State movsz eax, ax ; 4 byte sign extention of eax : FFFF CFFF 
    \State call print\underline{ }int ; prints -12289
   \end{algorithmic}
   \end{algorithm}
   \clearpage
   \item 1-byte quantities: E5 + 3A 
     \begin{center}
       \begin{tabular}{c c c}
       c &  & \\
        & E  & 5  \\
        +& 3   & A \\
        \hline 
      1 & 1 & F
       \end{tabular}
   \end{center}
   \begin{itemize}
       \item Set the Carry Flow into CF = 1
       \item Set the Over Flow into OF = 0
       \item Carry Flow because outcome is larger than a 1 byte 
       \item Overflow is 0 because a negative (E5) and positive (3A) summation does not lead to an overflow.
       \item Sign Extended: 0000 001F
       \item Since 1F is positive then 
       \begin{align*}
           1 \cdot 16 + 15 =\\
           31
       \end{align*}
                \begin{algorithm}
   \caption{macro prints}
   \begin{algorithmic}[1]
   \State mov al, 01Fh ; point al to 1F
    \State movsz eax, al ; eax : 4 byte sign extention  0000 001F 
    \State call print\underline{ }int ; prints 31
   \end{algorithmic}
   \end{algorithm}
   \end{itemize}
\end{enumerate}
\end{document}
